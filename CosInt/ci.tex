% Draft
%\documentclass[aps,prc,preprint,eqsecnum,groupedaddress,showpacs,floatfix,endfloats*]{revtex4}

% Good-lookin'
%\documentclass[aps,prc,preprint,eqsecnum,groupedaddress,showpacs,floatfix,tightenlines]{revtex4}
\documentclass[aps,prc,preprint,tightenlines]{revtex4}

% Two column
%\documentclass[aps,prc,twocolumn,eqsecnum,groupedaddress,showpacs,floatfix,tightenlines]{revtex4}

\usepackage{bm}
\newcommand{\sgn}{\mbox{sign}}

\begin{document}

\title{On the Properties of Certain Cosine Integrals}
\author{Jason R.\ Cooke}
\date{\today, \texttt{\jobname}}
\noaffiliation
\maketitle

The integral we wish to solve is
\begin{eqnarray}
I
&=& I(\{a_j\},\{n_j\},N,m) \nonumber \\
&=& \int_0^{2\pi} \frac{d\phi}{2\pi} e^{i m \phi} 
\prod_{j=1}^N \frac{1}{(1 + a_j \cos \phi)^{n_j}},
\end{eqnarray}
where $m$ is an integer, $n_j$ are a positive integers, and $a_j$ are
real numbers. Note that if one of the $a_j$'s is zero, then we can
reduce the integral to one with $N-1$ sets of parameters.

Some things to note about this integral:
\begin{enumerate}
\item{} \label{shifty} 
Since the range of integration is $2\pi$, and all the functions in the
integrand are periodic in that range, the range of integration can be
shifted by an arbitrary amount.
\item{} \label{swapy} 
When the variable of integration is changed from $\phi$ to $-\phi$,
the limits of integration can be rewritten as
\begin{eqnarray}
\int_0^{2\pi} d\phi \rightarrow 
\int_0^{-2\pi} d(-\phi) =
\int_{-2\pi}^0 d\phi =
\int_0^{2\pi} d\phi. 
\end{eqnarray}
Here we have used item number \ref{shifty}.  Since $\cos(-z)=\cos(z)$,
the only thing that changes in the integrand is the sign in the exponential.
Hence, $I(m)=I(-m)$, or the integral is only a function of the
magnitude of $m$.
\item{} The integral is real, since $I(m)^*=I(-m)=I(m)$.
\item{} We may write
\begin{eqnarray}
\frac{1}{(1+a\cos \phi)^n}
&=& \prod_{j=2}^n \left[ 1 + \frac{a}{j-1} \frac{d}{da} \right]
\frac{1}{1+a\cos \phi}
\end{eqnarray}

\end{enumerate}

so that
\begin{eqnarray}
&&\int_0^{2\pi} \frac{d\phi}{2\pi} e^{i \phi m}
\prod_{j=1}^N \frac{1}{(a_j + b_j \cos \phi)^{n_j}} \nonumber\\
&&\qquad =
\prod_{l=1}^N \frac{(-1)^{(n_l-1)}}{(n_l-1)!} \frac{d^{n_l-1}}{d a_l^{n_l-1}}
\int_0^{2\pi} \frac{d\phi}{2\pi} e^{i \phi m} 
\prod_{j=1}^N \frac{1}{a_j + b_j \cos \phi}. \label{a:ci:firstint}
\end{eqnarray}
Since $b_j\neq0$, we can factor out a
$\prod_{j=1}^N\frac{1}{b_j^{n_j}}$, and we relabel
$\frac{a_j}{b_j}\rightarrow a_j$. Then, the non-trivial integral is
\begin{eqnarray}
I(\{a_i\},m,N)
&=&
\int_0^{2\pi} \frac{d\phi}{2\pi} e^{i \phi |m|} 
\prod_{j=1}^N \frac{1}{a_j + \cos \phi} \nonumber \\
&=&
\int_0^{2\pi} \frac{d\phi}{2\pi} e^{i \phi |m|} 
\prod_{j=1}^N \frac{2 e^{i \phi}}
{e^{2 i \phi} + 2a_j e^{i \phi} + 1}.
\end{eqnarray}

Now we let $z=e^{i\phi}$, so $d\phi e^{i\phi}= dz/i$, which allows the
integral to be converted to a contour integral over the contour $C$, a
unit circle in the complex plane. Then,
\begin{eqnarray}
I(\{a_i\},m,N) 
&=&
\int_C \frac{dz}{2\pi i} z^{|m|-1}
\prod_{j=1}^N \frac{2 z}{z^2 + 2a_j z + 1} \nonumber \\
&=&
\int_C \frac{dz}{2\pi i} z^{|m|-1}
\prod_{j=1}^N \frac{2 z}{(z-z_{j+})(z-z_{j-})},
\end{eqnarray}
where
\begin{eqnarray}
z_{j\pm} = \left\{
\begin{array}{ll}
-a_j \pm \mbox{sign}(a_j) \sqrt{a_j^2 -  1 }
& \mbox{if } |a_j| > 1 \\
-a_j \mp i\sqrt{ 1  - a_j^2} = -e^{\pm i \phi_i)} 
& \mbox{if } |a_j|\leq 1 \\ 
\end{array} \right\}.
\end{eqnarray}
Here $\phi_i=\arccos \sqrt{1-a_i^2}$.

We will only consider the case where $|a_j|>1$. In this case, only the
only poles that contribute are the ones at $z=z_{j+}$.  The contour
integral is done by inspection to obtain
\begin{eqnarray}
I(\{a_i\},m,N) 
&=&
\sum_{k=1}^N
\int_{z_{k+}} \frac{dz}{2\pi i}
\frac{2 z^{|m|}}{(z-z_{k+})(z-z_{k-})}
\prod_{j=1,j\neq k}^N
\frac{2 z}{z^2 + 2 a_j z + 1} \\
&=&
\sum_{k=1}^N \frac{2 z_{k+}^{|m|}}{z_{k+}-z_{k-}}
\prod_{j=1,j\neq k}^N
\frac{2 z_{k+}}
{z_{k+}^2 + 2 a_j z_{k+} + 1}.
\end{eqnarray}
Note that
\begin{eqnarray}
z_{k\pm}^2 + 2 a_j z_{k\pm} + 1
&=& z_{k\pm}^2 + 2 a_k z_{k\pm} + 1
+ 2 (a_j -a_k) z_{k\pm}
= 2 (a_j -a_k) z_{k\pm}, \\
z_{k+}-z_{k-} &=& 
2\, \mbox{sign}(a_k) \sqrt{a_k^2-1}.
\end{eqnarray}
We define $c_k$ and $d_k$ by
\begin{eqnarray}
c_k &\equiv& \frac{z_{k+}-z_{k-}}{2}, \\
k_k &\equiv& z_{k+}^{|m|}.
\end{eqnarray}
With the appropriate replacements, we obtain
\begin{eqnarray}
I(\{a_i\},m,N) 
&=& \sum_{k=1}^N \frac{d_k}{c_k} \prod_{j=1,j\neq k}^N \frac{1}{a_j-a_k}.
\end{eqnarray}

Putting the $b_j$'s back in using the $a_i\rightarrow\frac{a_i}{b_i}$,
we get
\begin{eqnarray}
c_k &=& \mbox{sign}(a_k) \sqrt{a_k^2-b_k^2}, \\
d_k &=& \left(\frac{c_k-a_k}{b_k}\right)^{|m|}, \\
\int_0^{2\pi} \frac{d\phi}{2\pi} e^{i \phi m} 
\prod_{j=1}^N \frac{1}{a_j + b_j \cos\phi}
&=&
\sum_{k=1}^N \frac{d_k}{c_k}
\prod_{j=1,j\neq k}^N \frac{1}{a_j-a_k \frac{b_j}{b_k}}.
\end{eqnarray}
This expression is used in Eq.~(\ref{a:ci:firstint}) get
\begin{eqnarray}
&&\int_0^{2\pi} \frac{d\phi}{2\pi} e^{i \phi m} 
\prod_{j=1}^N \frac{1}{(a_j + b_j \cos\phi)^{n_j}} \nonumber\\
&& \qquad =
\prod_{l=1}^N \frac{(-1)^{(n_l-1)}}{(n_l-1)!} \frac{d^{n_l-1}}{d a_l^{n_l-1}}
\sum_{k=1}^N \frac{d_k}{c_k}
\prod_{j=1,j\neq k}^N \frac{1}{a_j-a_k \frac{b_j}{b_k}} \label{eqn:firstcosint}.
\end{eqnarray}

Specificly, for $N=1$, we can drop the subscripts.  For $n=1$,
\begin{eqnarray}
\int_0^{2\pi} \frac{d\phi}{2\pi} e^{i \phi m} 
\frac{1}{a + b \cos\phi}
&=& \frac{d}{c},
\end{eqnarray}
and for $n=2$,
\begin{eqnarray}
\int_0^{2\pi} \frac{d\phi}{2\pi} e^{i \phi m} 
\frac{1}{(a + b \cos\phi)^2}
&=& - \frac{d}{d a} \frac{d}{c}
= f_2 \frac{d}{c} \\
f_2 &=& \frac{a+|m| \, c}{c^2}.
\end{eqnarray}
Since $I(a,b,m)$ is like an eigenfunction of $\frac{d}{da}$, (but not
exactly, since the ``eigenvalue'' is a function of $a$), we an use
recursion to find that for general $n>2$,
\begin{eqnarray}
\int_0^{2\pi} \frac{d\phi}{2\pi} e^{i \phi m} 
\frac{1}{(a + b \cos\phi)^n}
&=& f_n I(a,b,m), \\
f_n &=& \frac{1}{n-1}\left( f_2 f_{n-1} - \frac{d}{da} f_{n-1} \right),
\end{eqnarray}
Using this we get
\begin{eqnarray}
f_1 &=& 1, \\
f_2 &=& \frac{a+|m| \, c}{c^2}, \\
f_3 &=& \frac{3 a^2 + c^2(m^2-1)+3a|m|c}{2c^4}.
\end{eqnarray}
The higher-order $f_i$'s can easily be found from analytic iteration
using a program like {\it Mathematica}, but apparently they do not have
a nice analytic form. We note that in general, $f_i$ is multiplied by a
factor of $c^{-2(i-1)}$.

We now check the equation for situations that may be numerically
unstable. First note that as $c\rightarrow0$, the $f$'s are singular,
which is due to the form of the original integral. Since the singular
part is multiplicative, it can easily be treated.

Instabilities appear when $b$ is very small. in which case we have to
take care of $d$ carefully as it appears to be singular. Let
$b=2a\delta$, so we can write
\begin{eqnarray*}
d_k &=& \left(\frac{\sqrt{1-4\delta^2}-1}{2\delta}\right)^{|m|} 
\approx \left[
-\delta
 ( 1 + \delta^2
 ( 1 + \delta^2
 ( 2 + \delta^2
 ( 5 + \delta^2
 (14 + \delta^2 \ldots ))))) \right]^{|m|}.
\end{eqnarray*}
This expression demonstrates that $d_k$ is not go as $b^{-|m|}$, but as
$b^{+|m|}$ as $b\rightarrow0$.

Now we analyze what happens where there is more than one term in
the denominator. We note first that
\begin{eqnarray}
\left[ \left( \frac{d}{dx} \right)^n f(x)g(x) \right]
&=& \sum_{m=0}^n \frac{n!}{m!(n-m)!}
\left[\left( \frac{d}{dx} \right)^m     f(x)\right]
\left[\left( \frac{d}{dx} \right)^{n-m} g(x)\right].
\end{eqnarray}
This means that Eq.~(\ref{eqn:firstcosint}) can be written as
\begin{eqnarray}
&&\int_0^{2\pi} \frac{d\phi}{2\pi} e^{i \phi m} 
\prod_{j=1}^N \frac{1}{(a_j + b_j \cos\phi)^{n_j}} \nonumber\\
&&\qquad =
\sum_{k=1}^N
\left[\prod_{l=1}^N \sum_{i_l=0}^{n_l-1} \right]
\left[
\frac{(-1)^{(n_l-1-i_l)}}{(n_l-1-i_l)!} \frac{d^{n_l-1-i_l}}{d a_l^{n_l-1-i_l}}
\left\{ \frac{d_k}{c_k} \right\}_{\text{on}}
\right]
\nonumber\\&&\qquad\qquad \times
\left[
\frac{(-1)^{(i_l)}}{(i_l)!} \frac{d^{i_l}}{d a_l^{i_l}}
\left\{
\prod_{j=1,j\neq k}^N \frac{1}{a_j-a_k \frac{b_j}{b_k}}
\right\}_{\text{on}}
\right].
\end{eqnarray}
The product can be broken into two parts, one where $l=k$ and another
where $l\neq k$.
\begin{eqnarray}
&&\int_0^{2\pi} \frac{d\phi}{2\pi} e^{i \phi m} 
\prod_{j=1}^N \frac{1}{(a_j + b_j \cos\phi)^{n_j}} \nonumber\\
&&\qquad=
\sum_{k=1}^N
\left[\prod_{l=1,l\neq k}^N \sum_{i_l=0}^{n_l-1} \right]
\left[
\frac{(-1)^{(n_l-1-i_l)}}{(n_l-1-i_l)!} \frac{d^{n_l-1-i_l}}{d a_l^{n_l-1-i_l}}
\left\{ \frac{d_k}{c_k} \right\}_{\text{on}}
\right]
\nonumber\\&&\qquad\qquad \times
\left[
\frac{(-1)^{(i_l)}}{(i_l)!} \frac{d^{i_l}}{d a_l^{i_l}}
\left\{
\prod_{j=1,j\neq k}^N \frac{1}{a_j-a_k \frac{b_j}{b_k}}
\right\}_{\text{on}}
\right]
\nonumber\\&&\qquad
\times 
\left[\sum_{i_k=0}^{n_k-1} \right]
\left[
\frac{(-1)^{(n_k-1-i_k)}}{(n_k-1-i_k)!} \frac{d^{n_k-1-i_k}}{d a_k^{n_k-1-i_k}}
\left\{ \frac{d_k}{c_k} \right\}_{\text{on}}
\right]
\nonumber\\&&\qquad\qquad \times
\left[
\frac{(-1)^{(i_k)}}{(i_k)!} \frac{d^{i_k}}{d a_k^{i_k}}
\left\{
\prod_{j=1,j\neq k}^N \frac{1}{a_j-a_k \frac{b_j}{b_k}}
\right\}_{\text{on}}
\right].
\end{eqnarray}
Now we note that $\frac{d}{da_l}\frac{d_k}{c_k}=0$, and so on, to get
\begin{eqnarray}
&&\int_0^{2\pi} \frac{d\phi}{2\pi} e^{i \phi m} 
\prod_{j=1}^N \frac{1}{(a_j + b_j \cos\phi)^{n_j}} \nonumber\\
&&\qquad=
\sum_{k=1}^N
\prod_{l=1,l\neq k}^N 
\left[
\frac{(-1)^{(n_l-1)}}{(n_l-1)!} \frac{d^{n_l-1}}{d a_l^{n_l-1}}
\left\{
\prod_{j=1,j\neq k}^N \frac{1}{a_j-a_k \frac{b_j}{b_k}}
\right\}_{\text{on}}
\right]
\nonumber\\&&\qquad\qquad\times 
\frac{d_k}{c_k} \sum_{i_k=0}^{n_k-1} f_{n_k-i_k}
\left[
\frac{(-1)^{i_k}}{i_k!} \frac{d^{i_k}}{d a_k^{i_k}}
\left\{
\prod_{j=1,j\neq k}^N \frac{1}{a_j-a_k \frac{b_j}{b_k}}
\right\}_{\text{on}}
\right]. \label{eq:cibigthing}
\end{eqnarray}
The first derivative term in Eq.~(\ref{eq:cibigthing}) can be written as
\begin{eqnarray}
&&\int_0^{2\pi} \frac{d\phi}{2\pi} e^{i \phi m} 
\prod_{j=1}^N \frac{1}{(a_j + b_j \cos\phi)^{n_j}} \nonumber\\
&&\qquad=
\sum_{k=1}^N \frac{d_k}{c_k} \sum_{i_k=0}^{n_k-1} f_{n_k-i_k}
\frac{(-1)^{i_k}}{i_k!} \frac{d^{i_k}}{d a_k^{i_k}}
\prod_{j=1,j\neq k}^N \frac{1}{\left(a_j-a_k
\frac{b_j}{b_k}\right)^{n_j}}.
\label{eq:ci:firstderterm}
\end{eqnarray}

Now, we consider the situation where all the $b_i$ have the same
value,$b$. In the case where $k=N$, the complicated part of
Eq.~(\ref{eq:ci:firstderterm}) is 
\begin{eqnarray}
h_N(i_N,N)
&\equiv& \frac{1}{i_N!} \frac{d^{i_N}}{d a_N^{i_N}}
\prod_{j=1}^{N-1} \frac{1}{(a_j-a_N)^{n_j}} \\
&=& \sum_{i_{N-1}=0}^{i_N} h_{N-1}(i_{N-1},N-1)
\frac{1}{(i_N-i_{N-1})!} 
\nonumber\\&&\qquad\times
\frac{d^{(i_N-i_{N-1})}}{d a_N^{(i_N-i_{N-1})}}
\frac{1}{(a_{N-1}-a_N)^{n_{N-1}}} \\
&=& \sum_{i_{N-1}=0}^{i_N} h_{N-1}(i_{N-1},N-1)
\frac{(i_N-i_{N-1}+  n_{N-1}-1)!}
     {(i_N-i_{N-1})!(n_{N-1}-1)!}
\nonumber\\&&\qquad\times
\frac{1}{(a_{N-1}-a_N)^{n_{N-1}+i_N-i_{N-1}}} \\
&=& \prod_{l=3}^N
\left[
\sum_{i_{l-1}=0}^{i_l}
\frac{(i_l-i_{l-1}+  n_{l-1}-1)!}
     {(i_l-i_{l-1})!(n_{l-1}-1)!} \frac{1}{(a_{l-1}-a_N)^{n_{l-1}+i_l-i_{l-1}}}
\right]
\\&&\times
\frac{(i_2+n_1-1)!}{i_2!(n_1-1)!} \frac{1}{(a_1-a_N)^{n_1+i_2}}.
\end{eqnarray}
From this, we can cyclically permute (or shuffle in any
order) the labels to get a similar expression for $k\neq N$.

For definiteness, we consider a few special cases. If $n_j=1$ for all
$j$ except $N$, we find that
\begin{eqnarray}
h_2(i,1) &=& \frac{\delta_{i,0}}{(a_2-a_1)^{n_2}}, \\
h_2(i,2) &=& \frac{1}{(a_1-a_2)^{i+1}},
\end{eqnarray}
and
\begin{eqnarray}
h_3(i,1) &=& \frac{\delta_{i,0}}{(a_2-a_1)(a_3-a_1)^{n_3}}, \\
h_3(i,2) &=& \frac{\delta_{i,0}}{(a_1-a_2)(a_3-a_2)^{n_3}}, \\
h_3(i,3) &=& \sum_{j=0}^i \frac{1}{(a_1-a_3)^{j+1}(a_2-a_3)^{i-j+1}}.
\end{eqnarray}
Using these we can write
\begin{eqnarray}
g_k &=& \sum_{i_k=0}^{n_k-1} (-1)^{i_k} f_{n_k-i_k} h_N(i_k,k), \\
\int_0^{2\pi} \frac{d\phi}{2\pi} e^{i \phi m} 
\prod_{j=1}^N \frac{1}{(a_j + b_j \cos\phi)^{n_j}}
&=& \sum_{k=1}^N \frac{d_k}{c_k} g_k.
\end{eqnarray}

\section*{Approximations}

For $N=2$, consider what happens when $a_1 \approx a_2$. We can write
$a_2 = a$, $a_1 = a - \delta$, so that
\begin{eqnarray}
&&I(a_1,a_2,b,n_1,n_2,m)\nonumber\\
&&\qquad= \int_0^{2\pi} \frac{d\phi}{2\pi} e^{i \phi m} 
\frac{1}{(a_2 + b \cos\phi - \delta)^{n_1}}
\frac{1}{(a_2 + b \cos\phi)^{n_2}} \\
&&\qquad= \int_0^{2\pi} \frac{d\phi}{2\pi} e^{i \phi m} 
\frac{1}{(a_2 + b \cos\phi)^{n_1+n_2}}
\sum_{j=0}^\infty \frac{(n_1-1+j)!}{(n_1-1)!j!}
\frac{\delta^j}{(a+b\cos\phi)^j} \\
&&\qquad= \left[ \sum_{j=0}^\infty (a_2-a_1)^j \frac{(n_1-1+j)!}{(n_1-1)!j!}
f_{n_1+n_2+j} \right] \frac{d_2}{c_2}.
\end{eqnarray}

For $N=3$, we could have $a_2\approx a_3$ (or equivalently
$a_1\approx a_3$). In this case,
\begin{eqnarray}
&&I(a_1,a_2,a_3,b,1,1,n,m) \nonumber\\
&&\qquad= \sum_{j=0}^\infty (a_3-a_2)^j
\int_0^{2\pi} \frac{d\phi}{2\pi} e^{i \phi m} 
\frac{1}{ a_1 + b \cos\phi}
\frac{1}{(a_3 + b \cos\phi)^{n+j+1}} \\
&&\qquad= \sum_{j=0}^\infty (a_3-a_2)^j I(a_1,a_3,b,1,n+j+1,m).
\end{eqnarray}
When $a_1\approx a_2$ but $a_1\not\approx a_3$, we get
\begin{eqnarray}
I(a_1,a_2,a_3,b,1,1,n,m)
&=& \sum_{j=0}^\infty (a_2-a_1)^j I(a_1,a_3,b,2+j,n,m).
\end{eqnarray}

\section*{Summary}

We define
\begin{eqnarray}
I(\{a_i\},b,\{n_i\},m) &=&
\int_0^{2\pi} \frac{d\phi}{2\pi} e^{i \phi m} 
\prod_i \frac{1}{(a_i+b\cos\phi)^{n_i}}, \\
c_k &=& a_k \sqrt{1-\left(\frac{b_k}{a_k}\right)^2}, \\
d_k &=& \left(\frac{c_k-a_k}{b_k}\right)^{|m|}.
\end{eqnarray}
We assume that $n_i>0$ and $b_i\neq0$.

For $N=1$, the integral is
\begin{eqnarray}
I(a,b,n,m) &=& f_n \frac{d}{c},
\end{eqnarray}
and for $N=2$, the integral is
\begin{eqnarray}
I(a_1,a_2,b,n_1,n_2,m)
&=&
(-1)^{n_2}
\left( \sum_{j=0}^{n_1-1} f_{n_1-j} \frac{(j+n_2-1)!}{j!(n_2-1)!}
\frac{1}{(a_1-a_2)^{j+n_2}} \right)
\frac{d_1}{c_1} \nonumber\\&&
+
(-1)^{n_1}
\left( \sum_{j=0}^{n_2-1} f_{n_2-j} \frac{(j+n_1-1)!}{j!(n_1-1)!}
\frac{1}{(a_2-a_1)^{j+n_1}} \right)
\frac{d_2}{c_2}.
\end{eqnarray}
For $N=2$, and with $n_1=1$ the integral simplifies to
\begin{eqnarray}
I(a_1,a_2,b,1,n,m)
&=& \frac{1}{(a_2-a_1)^n} \frac{d_1}{c_1}
- \left( \sum_{j=0}^{n-1} f_{n-j} \frac{1}{(a_2-a_1)^{j+1}} \right)
\frac{d_2}{c_2}.
\end{eqnarray}
When $N=3$ and $n_1=n_2=1$ the integral is
\begin{eqnarray}
&&I(a_1,a_2,a_3,b,1,1,n,m) \nonumber\\
&&\qquad=
\frac{1}{(a_2-a_1)(a_3-a_1)^n} \frac{d_1}{c_1} +
\frac{1}{(a_1-a_2)(a_3-a_2)^n} \frac{d_2}{c_2} \nonumber\\&& \qquad\qquad+
\left[ \sum_{i=0}^{n-1} (-1)^i f_{n-i}
\sum_{j=0}^i \frac{1}{(a_1-a_3)^{j+1}(a_2-a_3)^{i-j+1}} \right] \frac{d_3}{c_3}.
\end{eqnarray}

\section{Old}

To do this integral, we pull out complex analysis, and use this corollary
to the Cauchy-Gorsat theorem,
\begin{eqnarray}
f^{(n)}(z) &=& \frac{n!}{2\pi i} \int_C \frac{ds \, f(s)}{(s-z)^{n+1}}
 \qquad n \in \{0,1,2,3,\ldots\} \nonumber
\end{eqnarray}
where the $s$ is contained in the contour, and $f$ is analytic inside
the contour.  This can be rewritten as
\begin{eqnarray}
\int_C \frac{dz}{(z-s)^{n}} f(z)
&=& 
\frac{2\pi i}{(n-1)!} \frac{d^{n-1}}{dz^{n-1}} \left.f(z)\right|_{z=s} 
\nonumber 
\end{eqnarray}

Using this,
\begin{eqnarray}
I(a,b,l\geq 0,m,n) &=& 
\frac{2^{m+n}}{(m-1)!} \frac{d^{m-1}}{d z^{m-1}} \left(
\frac{z^{l+m+n-1}}{(z-z_{a-})^m (z^2 + 2 b z + 1)^n}
\right)_{z=z_{a+}} + \nonumber \\
& & \qquad +
\frac{2^{m+n}}{(n-1)!} \frac{d^{n-1}}{d z^{n-1}} \left(
\frac{z^{l+m+n-1}}{(z-z_{b-})^n (z^2 + 2 a z + 1)^m}
\right)_{z=z_{b+}} \nonumber
\end{eqnarray}

Specificly, for $m=1$ and $n=0$,
\begin{eqnarray}
I(a,b,l\geq 0,m=1,n=0) = 2 \frac{z_{a+}^l}{z_{a+}-z_{a-}}
= \frac{\sgn(a)}{\sqrt{a^2-1}} \left(-a + \sgn(a) \sqrt{a^2-1}\right)^l 
\nonumber
\end{eqnarray}
and for $m=2$, $n=0$,
\begin{eqnarray}
&&I(a,b,l\geq 0,m=2,n=0) \nonumber \\
&=& 4 \frac{z_{a+}^{l+1}}{(z_{a+}-z_{a-})^2} 
\left( \frac{l+1}{z_{a+}} - \frac{2}{z_{a+} - z_{a-}} \right) \nonumber \\
&=& \frac{1}{a^2-1} \left(-a + \sgn(a) \sqrt{a^2-1}\right)^{l+1}
\left( \frac{l+1}{-a + \sgn(a) \sqrt{a^2-1}} - \frac{\sgn(a)}{\sqrt{a^2-1}}
\right) \nonumber \\
&=& \frac{1}{a^2-1} \left(-a + \sgn(a) \sqrt{a^2-1}\right)^{l+1}
\left( 
\frac{|a| + l \sqrt{a^2-1}}{(-a + \sgn(a) \sqrt{a^2-1}) \sqrt{a^2-1}} 
\right) \nonumber \\
&=& \frac{1}{(a^2-1)^{3/2}} \left(-a + \sgn(a) \sqrt{a^2-1}\right)^l
\left(|a| + l \sqrt{a^2-1}\right) \nonumber \\
&=& I(a,b,l\geq 0,m=1,n=0) \frac{a + l \, \sgn(a) \sqrt{a^2-1}}{a^2-1}
\nonumber
\end{eqnarray}
and for $m=1$, $n=1$, we first note that
$z_{c\pm}^2 + 2 d z_{c\pm} + 1 
=z_{c\pm}^2 + 2 c z_{c\pm} + 1 + 2 (d-c) z_{c\pm} = 2 (d-c) z_{c\pm}$.
Also, it it happens that $a=b$, then we should use the formula above.
Then,
\begin{eqnarray}
&&I(a,b,l\geq 0,m=1,n=1)\nonumber \\
&=& 
4 \frac{z_{a+}^{l+1}}{(z_{a+}-z_{a-}) (z_{a+}^2 - 2 b z_{a+} + 1)} +
4 \frac{z_{b+}^{l+1}}{(z_{b+}-z_{b-}) (z_{b+}^2 - 2 a z_{a+} + 1)} 
\nonumber \\
&=& 
4 \frac{z_{a+}^{l+1}}{(2\, \sgn(a)\sqrt{a^2-1})(2 (b-a) z_{a+})} +
4 \frac{z_{b+}^{l+1}}{(2\, \sgn(b)\sqrt{b^2-1})(2 (a-b) z_{b+})} 
\nonumber \\
&=& \frac{1}{a-b} \left(
\frac{\left(-b+\sgn(b)\sqrt{b^2-1}\right)^l}{\sgn(b)\sqrt{b^2-1}} -
\frac{\left(-a+\sgn(a)\sqrt{a^2-1}\right)^l}{\sgn(a)\sqrt{a^2-1}} \right)
\nonumber
\end{eqnarray}

Now we wish to return to our original integral,
$I(a,b,l,m) = \frac{1}{b^m} I(a/b,l,m)$.  Therefore, we get
\begin{eqnarray}
\int_0^{2\pi} \frac{d\theta}{2\pi}
\frac{e^{i \theta l}}{(a + c \cos\theta)}
&=&
\frac{\sgn(a)}{\sqrt{a^2-c^2}} 
\left(\frac{-a + \sgn(a) \sqrt{a^2-c^2}}{c}\right)^{|l|}
\end{eqnarray}
if $a^2>c^2$.

And
\begin{eqnarray}
\int_0^{2\pi} \frac{d\theta}{2\pi}
\frac{e^{i \theta l}}{(a + c \cos\theta)^2}
&=& 
\left(\int_0^{2\pi} \frac{d\theta}{2\pi}
\frac{e^{i \theta l}}{(a + c \cos\theta)} \right)
\frac{a + l \, \sgn(a) \sqrt{a^2-c^2}}{a^2-c^2}
\end{eqnarray}

In particular, for the case when $a^2 > b^2$, and $a<0$ and $b>0$,
\begin{eqnarray}
I(a,b,l,1) &=& 
-\frac{2\pi}{\sqrt{a^2-b^2}} \left(\frac{-a - \sqrt{a^2-b^2}}{b}\right)^{|l|}
\label{cosintmeq1} \\
I(a,b,l,2) 
&=& 
 \frac{2\pi}{(a^2-b^2)^{3/2}} \left(\frac{-a - \sqrt{a^2-b^2}}{b}\right)^{|l|}
\left(-a + |l| \sqrt{a^2-b^2}\right) \label{cosintmeq2} \\
&=& I(a,b,l,1) \frac{a - |l| \sqrt{a^2-b^2}}{a^2-b^2} \nonumber 
\end{eqnarray}
Note that the first equation is negative definite, while the second one is 
positive definite.


\end{document}
